\subsection{Récupérer des jetons connexes d'un marché}
\label{linkedtokens}

Pour récupérer nb-jetons connexes d'un marché, on parcourt ce dernier du début à la fin avec un compteur de jetons connexes.

\noindent Différents cas de figure :
\begin{itemize}
    \item Si le pointeur n'est pas \code{NULL} (la case du marché contient un jeton) :
    \begin{itemize}
        \item Si le compteur est égal à nb alors on ajoute l'indice à une liste sans incrémenter le compteur.
        \item Sinon on incrémente le compteur de 1
    \end{itemize}
    \item Si le pointeur est \code{NULL} (la case du marché est vide) :
    \begin{itemize}
        \item On remet le compteur à 0.
    \end{itemize}
\end{itemize}

Ainsi à l'issue de la boucle on peut retourner un indice présent dans la liste pris de manière aléatoire. Si la liste est vide, on retourne -1.

La complexité de cet algorithme est $\theta(n)$, avec n le nombre de jetons dans la partie.

\begin{summary}
Ici on considère des jetons connexes lorsqu'ils sont reliés par le chemin continu dessiné par le plateau (cf \ref{fig:board_example}) et non lorsqu'ils sont voisins sur le plateau. C'est un choix assumé.
\end{summary}